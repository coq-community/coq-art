\documentclass[a4]{report}
\usepackage[utf8x]{inputenc}

\usepackage{fontspec}
%\usepackage{multind}
\usepackage{alltt}   
\usepackage{verbatim}
\usepackage{amssymb}
\usepackage{amsmath}
\usepackage{theorem}
\usepackage[dvips]{epsfig}
%\usepackage{epic}
%\usepackage{eepic}
\usepackage{pict2e}
\usepackage{qtree}
\usepackage{moreverb} 
\usepackage{color}
\usepackage{pifont}
\usepackage{xr}
\usepackage{url}
\usepackage{varioref}
\usepackage[note]{marginote} 
%[nonote] pour supprimer toutes les notes de marge
\usepackage{natbib}
\usepackage{hyperref}


\newcommand{\variantspringer}[1]{#1}
\newcommand{\marginok}[1]{\marginpar{\raggedright OK:#1}}
\newcommand{\tab}{{\null\hskip1cm}}
\newcommand{\Ltac}{\mbox{\emph{$\cal L$}tac}}
\newcommand{\coq}{\mbox{\emph{Coq}}}
\newcommand{\lcf}{\mbox{\emph{LCF}}}
\newcommand{\hol}{\mbox{\emph{HOL}}}
\newcommand{\pvs}{\mbox{\emph{PVS}}}
\newcommand{\isabelle}{\mbox{\emph{Isabelle}}}
\newcommand{\prolog}{\mbox{\emph{Prolog}}}
\newcommand{\goalbar}{\tt{}============================\it}
\newcommand{\gallina}{\mbox{\emph{Gallina}}}
\newcommand{\joker}{\texttt{\_}}
\newcommand{\eprime}{\(\e^{\prime}\)}
\newcommand{\Ztype}{\citecoq{Z}}
\newcommand{\propsort}{\citecoq{Prop}}
\newcommand{\setsort}{\citecoq{Set}}
\newcommand{\typesort}{\citecoq{Type}}
\newcommand{\ocaml}{\mbox{\emph{OCAML}}}
\newcommand{\haskell}{\mbox{\emph{Haskell}}}
\newcommand{\why}{\mbox{\emph{Why}}}
\newcommand{\Pascal}{\mbox{\emph{Pascal}}}

\newcommand{\ml}{\mbox{\emph{ML}}}

\newcommand{\scheme}{\mbox{\emph{Scheme}}}
\newcommand{\lisp}{\mbox{\emph{Lisp}}}

\newcommand{\implarrow}{\mbox{$\Rightarrow$}}
\newcommand{\metavar}[1]{?#1}
\newcommand{\notincoq}[1]{#1}
\newcommand{\coqscope}[1]{\%#1}
\newcommand{\arrow}{\mbox{$\rightarrow$}}
\newcommand{\fleche}{\arrow}
\newcommand{\funarrow}{\mbox{$\Rightarrow$}}
\newcommand{\ltacarrow}{\funarrow}
\newcommand{\coqand}{\mbox{\(\wedge\)}}
\newcommand{\coqor}{\mbox{\(\vee\)}}
\newcommand{\coqnot}{\mbox{\(\neg\)}}
\newcommand{\hide}[1]{}
\newcommand{\hidedots}[1]{...}
\newcommand{\sig}[3]{\texttt{\{}#1\texttt{:}#2 \texttt{|} #3\texttt{\}}}
\renewcommand{\neg}{\sim}
%\renewcommand{\marginpar}[1]{}

\addtocounter{secnumdepth}{1}
\providecommand{\og}{´}
\providecommand{\fg}{ª}
\definecolor{light}{rgb}{0.9,0.9,0.9}

\newcommand{\hard}{\mbox{\small *}}
\newcommand{\xhard}{\mbox{\small **}}
\newcommand{\xxhard}{\mbox{\small ***}}

%%% Operateurs, etc.
\newcommand{\impl}{\mbox{$\rightarrow$}}
\newcommand{\appli}[2]{\mbox{\tt{#1 #2}}}
\newcommand{\applis}[1]{\mbox{\texttt{#1}}}
\newcommand{\abst}[3]{\mbox{\tt{fun #1:#2 \funarrow #3}}}
\newcommand{\coqle}{\mbox{$\leq$}}
\newcommand{\coqge}{\mbox{$\geq$}}
\newcommand{\coqdiff}{\mbox{$\neq$}}
\newcommand{\coqiff}{\mbox{$\leftrightarrow$}}
\newcommand{\prodsym}{\mbox{\(\forall\,\)}}
\newcommand{\exsym}{\mbox{\(\exists\,\)}}

\newcommand{\substsign}{/}
\newcommand{\subst}[3]{\mbox{#1\{#2\substsign{}#3\}}}
\newcommand{\anoabst}[2]{\mbox{\tt[#1]#2}}
\newcommand{\letin}[3]{\mbox{\tt let #1:=#2 in #3}}
\newcommand{\prodep}[3]{\mbox{\tt \(\forall\,\)#1:#2,$\,$#3}}
\newcommand{\prodplus}[2]{\mbox{\tt\(\forall\,\)$\,$#1,$\,$#2}}
\newcommand{\dom}[1]{\textrm{dom}(#1)} % domaine d'un contexte (log function)
\newcommand{\norm}[1]{\textrm{n}(#1)} % forme normale (log function)
\newcommand{\coqZ}[1]{\mbox{\tt{`#1`}}}
\newcommand{\coqnat}[1]{\mbox{\tt{#1}}}
\newcommand{\coqcart}[2]{\mbox{\tt{#1*#2}}}
\newcommand{\alphacong}{\mbox{$\,\cong_{\alpha}\,$}} % alpha-congruence
\newcommand{\betareduc}{\mbox{$\,\rightsquigarrow_{\!\beta}$}\,} % beta reduction
%\newcommand{\betastar}{\mbox{$\,\Rightarrow_{\!\beta}^{*}\,$}} % beta reduction
\newcommand{\deltareduc}{\mbox{$\,\rightsquigarrow_{\!\delta}$}\,} % delta reduction
\newcommand{\dbreduc}{\mbox{$\,\rightsquigarrow_{\!\delta\beta}$}\,} % delta,beta reduction
\newcommand{\ireduc}{\mbox{$\,\rightsquigarrow_{\!\iota}$}\,} % delta,beta reduction


% jugement de typage
\newcommand{\these}{\boldsymbol{\large \vdash}}
\newcommand{\disj}{\mbox{$\backslash/$}}
\newcommand{\conj}{\mbox{$/\backslash$}}
%\newcommand{\juge}[3]{\mbox{$#1 \boldsymbol{\vdash} #2 : #3 $}}
\newcommand{\juge}[4]{\mbox{$#1,#2 \these #3 \boldsymbol{:} #4 $}}
\newcommand{\smalljuge}[3]{\mbox{$#1 \these #2 \boldsymbol{:} #3 $}}
\newcommand{\goal}[3]{\mbox{$#1,#2 \these^{\!\!\!?} #3  $}}
\newcommand{\sgoal}[2]{\mbox{$#1\these^{\!\!\!\!?} #2 $}}
\newcommand{\reduc}[5]{\mbox{$#1,#2 \these #3 \rhd_{#4}#5 $}}
\newcommand{\convert}[5]{\mbox{$#1,#2 \these #3 =_{#4}#5 $}}
\newcommand{\convorder}[5]{\mbox{$#1,#2 \these #3\leq _{#4}#5 $}}
\newcommand{\wouff}[2]{\mbox{$\emph{WF}(#1)[#2]$}}


%\newcommand{\mthese}{\underset{M}{\vdash}}
\newcommand{\mthese}{\boldsymbol{\vdash}_{\!\!M}}
\newcommand{\type}{\boldsymbol{:}}

% jugement absolu

%\newcommand{\ajuge}[2]{\mbox{$ \boldsymbol{\vdash} #1 : #2 $}}
\newcommand{\ajuge}[2]{\mbox{$\these #1 \boldsymbol{:} #2 $}}

%%% logique minimale
\newcommand{\propzero}{\mbox{$P_0$}} % types de Fzero

%%% logique propositionnelle classique
\newcommand {\ff}{\boldsymbol{f}} % faux
\newcommand {\vv}{\boldsymbol{t}} % vrai

\newcommand{\verite}{\mbox{$\cal{B}$}} % {\ff,\vv}
\newcommand{\sequ}[2]{\mbox{$#1 \vdash #2 $}} % sequent
\newcommand{\strip}[1]{#1^o} % enlever les variables d'un contexte



%%% tactiques
\newcommand{\decomp}{\delta} % decomposition
\newcommand{\recomp}{\rho} % recomposition

%%% divers
\newcommand{\cqfd}{\mbox{\textbf{cqfd}}}
\newcommand{\fail}{\mbox{\textbf{F}}}
\newcommand{\succes}{\mbox{$\blacksquare$}}
%%% Environnements


%% Fzero
\newcommand{\con}{\mbox{$\cal C$}}
\newcommand{\var}{\mbox{$\cal V$}}

\newcommand{\atomzero}{\mbox{${\cal A}_0$}} % types de base de Fzero
\newcommand{\typezero}{\mbox{${\cal T}_0$}} % types de Fzero
\newcommand{\termzero}{\mbox{$\Lambda_0$}} % termes de Fzero 
\newcommand{\conzero}{\mbox{$\cal C_0$}} % contextes de Fzero 

\newcommand{\buts}{\mbox{$\cal B$}} % buts

\definecolor{metavarcolor}{rgb}{0.5,0.0,1.0}
\definecolor{darkgreen}{rgb}{0.1,0.7,0.1}
\definecolor{answercolor}{rgb}{1.0,0.0,0.0}
\definecolor{normalcolor}{rgb}{0.0,0.0,0.0}
\definecolor{exbluecolor}{rgb}{0.1,0.1,0.9}
\definecolor{dontlookcolor}{rgb}{0.5,0.5,0.5}
\definecolor{termcolor}{rgb}{0.0,0.1,0.9}
\definecolor{lookcolor}{rgb}{0.8,0.2,0.0}
\definecolor{prooftermcolor}{rgb}{0.3,0.1,1.0}
\definecolor{typecolor}{rgb}{1.0,0.6,0.0}
\definecolor{taccolor}{rgb}{0.70,0.10,0.0}
\definecolor{pink}{rgb}{0.8,0.6,0.6}
\definecolor{darkmagenta}{rgb}{0.4,0.0,0.6}
\definecolor{darkblue}{rgb}{0.0,0.0,0.6}
\definecolor{darkgray}{rgb}{0.4,0.4,0.4}
%%%% scripts coq
\newcommand{\prompt}{\mbox{\sl Coq $<\;$}}
\newcommand{\natquicksort}{\texttt{nat\_quicksort}}
\newcommand{\citecoq}[1]{\mbox{\texttt{#1}}}
\newcommand{\safeit}{\it}
\newtheorem{exercice}{Exercise}[chapter]
\newtheorem{remarque}{Remark}[chapter]
\newtheorem{definition}{Definition}[chapter]
\newenvironment{newversion}{%
  \begin{trivlist}\item \color{red}\textbf{2012:}}%
  {\end{trivlist}}

\title{A Gentle Introduction to Type Classes and Relations in \coq{}}
\author{Pierre CastÈran\\Univ. Bordeaux, LaBRI, UMR 5800, F-33400 Talence, France.\\
CNRS, LaBRI, UMR 5800, F-33400 Talence, France.\\
\\
Matthieu Sozeau\\INRIA Paris \& PPS, Univ. Paris Diderot}
\begin{document}
\maketitle

\chapter{Introduction}

This document aims to give a small introduction to two important and
rather new features of the {\coq} proof assistant~\citep{Coq, coqart} : type classes and user defined
relations. We will mainly use examples for introducing these concepts, and 
the reader should consult \coq's reference manual for further information.

Type classes are presented in Chapter 18 ``Type Classes''  of {\coq}'s reference manual, and user defined relations in Chapter 25.

The complete definitons and proof scripts are structured as follows:
\begin{description}
\item[Power\_mono.v] Sadly monomorphic definitions of function
   $x\in \mathbb{Z},\;n\in\mathbb{N}\mapsto x^n$ (naÔve and binary algorithm).
\item[Matrices.v] A data type for $2\times 2$ matrices.
\item[Monoid.v] The \texttt{Monoid} type class; proof of equivalence of
  the naÔve and the binary power functions in any monoÔd. The abelian
  (a.k.a. commutative) monoid structure.
\item[Monoid\_op\_classes.v]
  Same contents as \texttt{Monoid.v}, but with operational type classes.
\item[Alternate\_defs.v] Dicussion about various possible representations of
        the monoid mathematical structure.
\item[Lost\_in\_NY.v] An introduction to user defined relations, and rewriting.
\item[EMonoid.v]  Monoids with respect to an equivalence relation.
\item[Trace\_Monoid.v] The trace monoid (a.k.a. partial commutation monoid).
\end{description}

A tar file of these scripts (for Coq V8.4) is available 
at the following address:

\href{http://www.labri.fr/perso/casteran/CoqArt/TypeClassesTut/src-V8.4.tar.gz}{http://www.labri.fr/perso/casteran/CoqArt/TypeClassesTut/src-V8.4.tar.gz}

\section{A little history}

Type classes were first introduced in {\haskell} by \citet{WadlerBlott89} and the {\isabelle} proof
assistant \citep{DBLP:conf/fpca/NipkowS91} at the begining of the
90's, to handle ad-hoc polymorphism
elegantly. \cite{DBLP:conf/tphol/SozeauO08} adapted the design to
{\coq}, making use of the powerful type system of {\coq} to integrate a
lightweight version of type classes which we'll introduce here.

The purpose of type class systems is to allow a form of generic
programming for a \emph{class} of types. Classes are presented as an
interface comprised of functions specific to a type, and in the case of
{\isabelle} and {\coq}, proofs about those functions as well. One can
write programs that are polymorphic over any type which has an instance
of the class, and later on, instantiate those to a specific type and
implementation (called \emph{instance}) of the class. Polymorphism is
ad-hoc as opposed to the parametric polymorphism found in e.g. ML:
while parametrically polymorphic functions behave uniformly for any
instantiation of the type, ad-hoc polymorphic functions have
type-specific behaviors. 

\chapter{An Introductory Example: Computing $x^n$}
\section{A Monomorphic Introduction}
Let us consider a simple arithmetic operation: raising some integer $x$ to

the $n$-th power, where $n$ is a natural number.

The following function definition is a direct translation of the mathematical
concept:
\begin{alltt}
Require Import ZArith.
Open Scope Z_scope.

Fixpoint power (x:Z)(n:nat) :=
  match n with 0\%nat => 1
             | S p =>  x * power x p
  end.

Compute power 2 40.\it\color{answercolor}
    =  1099511627776
     : Z
\end{alltt}

This definition can be considered as a very na\"{\i}ve way of programming,
since computing $x^n$ requires $n$ multiplications. 
 Nevertheless, this definition is 
very simple to read, and everyone can admit that it is  correct 
with respect to the mathematical definition of raising $x$ to the $n$-th power.
Thus, we can consider it as a \emph{specification}: when we write 
more efficient but less readable functions for exponentiation,
% \cite{DBLP:journals/ita/BrlekCHM95},
 we should be able
to prove their correctness by proving in \coq{} their equivalence\footnote{\emph{i.e.} extensional equality} 
with the na\"{\i}ve \texttt{power} function.

For instance the following function allows one to compute $x^n$, 
with a number of multiplications proportional to $\textrm{log}_2(n)$:


  \begin{alltt}
{Function} binary_power_mult (acc x:Z){(n:nat)}
                \{measure (fun i=>i) n\} : Z
{(* acc * (power x n) *)} :=
 match {{n}} with 0\%nat => acc
             | _ => if Even.even_odd_dec n
                    then  binary_power_mult  
                          acc (x * x) {(div2 n)}
                    else  binary_power_mult 
                         (acc * x) (x * x) {(div2 n)}
  end.{
Proof.
  intros;apply lt_div2; auto with arith.
  intros;apply lt_div2; auto with arith.}
Defined.

Definition binary_power (x:Z)(n:nat) := 
        binary_power_mult 1 x n.

Compute binary_power 2 40.\color{answercolor}\it
1099511627776: Z
\end{alltt}

We want now to \emph{prove} \texttt{binary\_power}'s correctness, \emph{i.-e.}
that this function and the naÔve \texttt{power} function are pointwise
equivalent.

Proving this equivalence in {\coq} may require a lot of work. 
Thus it is not worth at all writing  a proof dedicated only to powers of
integers. In fact, the correctness of \texttt{binary\_power} with respect to
\texttt{power} holds in any structure composed of an associative binary 
operation on some†domain, that admits a neutral element.
For instance, we can compute powers of square matrices  using the most efficient of both algorithms.


Thus, let us throw away our previous definition, and try do define them
in a more generic framework.

\begin{alltt}
Reset power. 
\end{alltt}


\section{On Monoids}\label{on-monoids}

\begin{definition}
A \emph{monoid} is a mathematical structure composed of:
\begin{itemize}
\item A carrier $A$ 
\item A binary, associative  operation $\circ$ on $A$
\item A neutral element $1\in A$ for $\circ$
\end{itemize}
\end{definition}

Such a mathematical structure can be defined in {\coq} as
a \emph{type class}~\citep{DBLP:conf/tphol/SozeauO08, math-classes, DBLP:journals/corr/abs-1106-3448}. In the following definition, parameterized by
a type $A$ (implicit), a binary operation \texttt{dot} and a neutral element 
\texttt{unit}, three fields describe the properties that \texttt{dot} and
\texttt{unit} must satisfy.


\begin{alltt}
Class Monoid \{A:Type\}(dot : A -> A -> A)(one : A) : Prop := \{
 dot_assoc : forall x y z:A, 
     dot x (dot y z) = dot (dot x y) z;
 unit_left : forall x, dot one x = x;
 unit_right : forall x, dot x one = x \}.
\end{alltt}


Note that other definitions could have been given for representing this mathematical structure. See Sect~\ref{alternate} for more details.


From an implementational point of view, such a type class is just a record type,
\emph{i.e.} an inductive type with a single constructor \texttt{Build\_Monoid}.

\begin{alltt}
Print Monoid.\it\color{red}

Record Monoid (A : Type) (dot : A -> A -> A) (one : A) : Prop := Build_Monoid
  \{ dot_assoc : forall x y z : A, dot x (dot y z) = dot (dot x y) z;
    one_left : forall x : A, dot one x = x;
    one_right : forall x : A, dot x one = x \}

For Monoid: Argument A is implicit and maximally inserted
For Build_Monoid: Argument A is implicit
For Monoid: Argument scopes are [type_scope _ _]
For Build_Monoid: Argument scopes are [type_scope _ _ _ _ _]
\end{alltt}
\label{ClassVsRecords}
Nevertheless, the implementation of type classes by M. Sozeau provides several specific tools 
---dedicated tactics for instance ---, and we advise the reader not to replace
the \texttt{Class} keyword with \texttt{Record} or \texttt{Inductive}.


With the command \texttt{About}, we can see the polymorphic type of
the fields of the class \texttt{Monoid}:

\begin{alltt}
About one_left.\it\color{red}
one_left :
forall (A : Type) (dot : A -> A -> A) (one : A),
Monoid dot one -> forall x : A, dot one x = x

Arguments A, dot, one, Monoid are implicit and maximally inserted
Argument scopes are [type_scope _ _ _ _]
one_left is transparent  
\end{alltt}

\subsection{Classes and Instances}
Members of a given class are called \emph{instances} of this class.
Instances are defined to the {\coq} system through the \texttt{Instance} keyword. Our first example is a definition of the monoid structure on the set
$\mathbb{Z}$ of integers, provided with integer multiplication, with $1$ as 
a neutral element. Thus we give these parameters to the \texttt{Monoid} class
(note that \texttt{Z} is implicitely given).

\begin{alltt}
Instance ZMult : Monoid  Zmult 1.
\end{alltt}

For this instance to be created, we need to prove that the binary operation
\texttt{Zmult} is associative and admits $1$†as neutral element.
Applying the constructor \texttt{Build\_Monoid} --- for instance with the tactic \texttt{split} --- generates three subgoals.
 

\begin{alltt}
split.\it\color{answercolor}
3 subgoals
  ============================
   forall x y z : Z, x * (y * z) = x * y * z

subgoal 2 is:
 forall x : Z, 1 * x = x
subgoal 3 is:
 forall x : Z, x * 1 = x\tt\color{black}
\end{alltt}

 Each subgoal is easily solved by \textcolor{taccolor}{intros; ring}.

When the proof is finished, we register our instance with a simple \texttt{Qed}.
Note that we used \texttt{Qed} because we consider a class of sort \texttt{Prop}. In some cases where instances must store some informative constants, ending
an instance construction with \texttt{Defined} may be necessary.

%We can look at the type of   \texttt{ZMult}:

\begin{alltt}
Check Zmult.\it\color{red}
ZMult : Monoid Zmult 1
\end{alltt}

 We explained \vpageref{ClassVsRecords} why it is better to use
the \texttt{Class} keyword than \texttt{Record} or \texttt{Inductive}.
For the same reason, the definition of an instance of some class should
be written using \texttt{Instance} and not \texttt{Lemma}, \texttt{Theorem},
\texttt{Example}, etc., nor \texttt{Definition}.

\subsection{A generic definition of \texttt{power}}

We are now able to give a definition of the function \texttt{power} that
can be applied with any instance of class \texttt{Monoid}:

A first definition could be:

\begin{alltt}
Fixpoint power \{A:Type\}\{dot:A->A->A\}\{one:A\}\{M: Monoid  dot one\}
               (a:A)(n:nat) :=
  match n with 0%nat => one
             | S p => dot a (power a p)
  end.

Compute power 2 10.\it\color{red}
= 1024 : Z
\end{alltt}

Happily, we can make the declaration of the three first  arguments implicit, by using the
\texttt{Generalizable Variables} command:

\begin{alltt}
Reset power.

Generalizable Variables A dot one.

Fixpoint power `\{M: Monoid A dot one\}(a:A)(n:nat) :=
  match n with 0%nat => one
             | S p => dot a (power a p)
  end.

Compute power 2 10.\it\color{red}
= 1024 : Z
\end{alltt}

The variables \texttt{A dot one} appearing in the binder for \texttt{M}
are implicitly bound before the binder for \texttt{M} and their types
are infered from the \texttt{Monoid A dot one} type. This syntactic sugar
helps abbreviate bindings for classes with parameters. The resulting
internal {\coq} term is exactly the same as the first definition above.

\subsection{Instance Resolution}

The attentive reader has certainly noticed that in the last computation, the
the binary operation \texttt{Zmult} and the neutral element $1$ need not to be
given explicitely\footnote{This is quite different from the basic implicit arguments mechanism, already able to infer the type \texttt{Z} from the type of $2$.}.
The mechanism that allows {\coq} to infer all the arguments needed by the \texttt{power} function to be applied is called \emph{instance resolution}.



In order to understand how it operates, let's have a look at \texttt{power}'s type:
  \begin{alltt}
About power.\it\color{answercolor}
power :
forall (A : Type) (dot : A -> A -> A) (one : A),
Monoid dot one -> A -> nat -> A

Arguments A, dot, one, M are implicit and maximally inserted
\tt\color{black}
Compute power 2 100.\it\color{answercolor}
 = 1267650600228229401496703205376 : Z\tt\color{black}

Set Printing Implicit.
Check power 2 100.\it\color{answercolor}
@power Z Zmult 1 {\color{blue}ZMult}  2 100 : Z \tt\color{black}
Unset Printing Implicit.
\end{alltt}

We see that the \emph{instance} \texttt{ZMult} has been 
inferred from the type of $2$. We are in the simple case where only one monoid
of carrier \texttt{Z} has been declared as an instance of the \texttt{Monoid} class.

The implementation of type classes in {\coq} can retrieve the instance \texttt{ZMult}
from the type \texttt{Z}, then filling the arguments   \texttt{Zmult} and $1$ from
\texttt{ZMult}'s definition.

       

\subsection{More Monoids}
\subsubsection{Matrices over some ring}\label{matring}
We all know that multiplication of square matrices is associative and admits
identity matrices as neutral elements. For simplicity's sake let us restrict our study
to $2 \times 2$ matrices over some ring.

We first load the \texttt{Ring} library, the open a section with some useful declarations and notations. The reader may consult \coq's documentation about the \texttt{Ring} library.

\begin{alltt}
Require Import Ring.

Section matrices.
 Variables (A:Type)
           (zero one : A) 
           (plus mult minus : A -> A -> A)
           (sym : A -> A).
 Notation "0" := zero.  Notation "1" := one.
 Notation "x + y" := (plus x y).  
 Notation "x * y " := (mult x y).

 Variable rt : ring_theory  zero one plus mult minus sym (@eq A).

 Add Ring Aring : rt.  
\end{alltt}


We can now define a carrier type for $2\times 2$-matrices, as well as matrix multiplication and the identity matrix:

\begin{alltt}
Structure M2 : Type := \{c00 : A;  c01 : A;
                        c10 : A;  c11 : A\}.

Definition Id2 : M2 := Build_M2 1 0 0 1.

Definition M2_mult (m m':M2) : M2 :=
 Build_M2 (c00 m * c00 m' + c01 m * c10 m')
          (c00 m * c01 m' + c01 m * c11 m')
          (c10 m * c00 m' + c11 m * c10 m')
          (c10 m * c01 m' + c11 m * c11 m').
\end{alltt}

As for multiplication of integers, we can now define an instance of 
\texttt{Monoid} for the type \texttt{M2}.

\begin{alltt}
Global Instance M2_Monoid : Monoid M2_mult Id2. \it\color{red}
(*  Proof skipped *) \tt\color{black}
Qed.

End matrices.
\end{alltt}


We want now to play with $2\times 2$ matrices over $\mathbb{Z}$.
We declare an instance \texttt{M2Z} for this purpose, and can use directly
the function \texttt{power}.

\begin{alltt}
Instance M2Z : Monoid  _ _ := M2_Monoid Zth.

Compute power (Build_M2 1 1 1 0) 40.\it\color{red}
   = \{|
       c00 := 165580141;
       c01 := 102334155;
       c10 := 102334155;
       c11 := 63245986 |\}
     : M2 Z\tt\color{black}

Definition fibonacci (n:nat) :=
  c00 (power (Build_M2 1 1 1 0) n).

Compute fibonacci 20.\it\color{red}
= 10946
 :Z
\end{alltt}


\section{Reasoning within a Type Class}

We are now able to consider again the equivalence between two functions for computing
powers.
Let us define the binary algorithm for any monoid:

First, we define an auxiliary function. We use the \texttt{Program}
extension to define an efficient version of exponentiation using an
accumulator. The function is defined by well-founded recursion on the
exponent \texttt{n}:

% \begin{alltt}
% Program
% Fixpoint binary_power_mult (A:Type) (dot:A->A->A) (one:A)
%                            (M: @Monoid A dot one)
%      (acc x:A)(n:nat){measure  n} : A 
%   (* acc * (x ** n) *) :=
%   match n with 0%nat => acc
%              | _ => 
%               if  Even.even_odd_dec n
%               then binary_power_mult  _   acc (dot x x) (div2 n)
%               else binary_power_mult   _ (dot acc  x) (dot  x  x) (div2 n)
%   end.


% Solve Obligations using program_simpl; intros; apply lt_div2; auto with arith.
% \end{alltt}
\begin{alltt}
Function binary_power_mult (A:Type) (dot:A->A->A) (one:A) (M: @Monoid A dot one)
    (acc x:A)(n:nat)\{measure (fun i=>i) n\} : A 
  (* acc * (x ** n) *) :=
  match n with
  | 0\%nat => acc
  | _ => if  Even.even_odd_dec n
          then binary_power_mult  _   acc (dot x x) (div2 n)
          else binary_power_mult   _ (dot acc  x) (dot  x  x) (div2 n)
  end.
intros;apply lt_div2; auto with arith.
intros; apply lt_div2; auto with arith.
Defined.

Definition binary_power `\{M:Monoid\} x n := binary_power_mult M one x n.

Compute binary_power 2 100.\it\color{red}
= 1267650600228229401496703205376
     : Z
\end{alltt}

\subsection{The Equivalence Proof}\label{equiv-proof}

The proof of equivalence between \texttt{power} and \texttt{binary\_power} is quite
long, and can be split in several lemmas. 
Thus, it is useful to open a section, in which we fix some arbitrary monoid $M$.
Such a declaration is made with the \texttt{Context} command, which can be considered
as a version of \texttt{Variables} for declaring arbitrary instances of a given class.

\begin{alltt}
Section About_power.

Require Import Arith.
  Context `\{M:Monoid A dot one\}.
\end{alltt}

It is a good practice to define locally some specialized notations and tactics.

\begin{alltt}
  Ltac monoid_rw :=
    rewrite (@one_left A dot one M) || 
    rewrite (@one_right A dot one M)|| 
    rewrite (@dot_assoc A dot one M).

  Ltac monoid_simpl := repeat monoid_rw.

  Local Infix "*" := dot.
  Local Infix "**" := power (at level 30, no associativity).
\end{alltt}

\subsection{Some Useful Lemmas About power}

We start by proving some well-known equalities about powers in a monoid.
Some of these equalities are integrated later in simplification tactics. 

\begin{alltt}
  Lemma power_x_plus : forall x n p, x ** (n + p) =  x ** n *  x ** p.
  Proof.
    induction n as [| p IHp];simpl.
     intros; monoid_simpl;trivial.
    intro q;rewrite (IHp q);  monoid_simpl;trivial. 
  Qed.

  Ltac power_simpl := repeat (monoid_rw || rewrite <- power_x_plus).

  Lemma power_commute : forall x n p,  
               x ** n * x ** p = x ** p * x ** n. 
  {\it\color{red}(* Proof skipped *)}

  Lemma power_commute_with_x : forall x n ,  
        x * x ** n = x ** n * x.
  {\it\color{red} (* Proof skipped *)}                            
 
  Lemma power_of_power : forall x n p,  (x ** n) ** p = x ** (p * n).
  {\it\color{red} (* Proof skipped *) }     

  Lemma power_S : forall x n, x *  x ** n = x ** S n.
  {\it\color{red} (* Proof skipped *)} 

  Lemma sqr : forall x, x ** 2 =  x * x.
  {\it\color{red} (* Proof skipped *)}

  Ltac factorize := repeat (
                rewrite <- power_commute_with_x ||
                rewrite  <- power_x_plus  ||
                rewrite <- sqr ||
                rewrite power_S ||
                rewrite power_of_power).

  Lemma power_of_square : forall x n, (x * x) ** n = x ** n * x ** n.
  {\it\color{red}(* Proof skipped *)}
\end{alltt}

\subsection{Final Steps}
We are now able to prove that the auxiliary function \texttt{binary\_power\_mult}
satiisfies its intuitive meaning. The proof --- partly skipped --- uses well-founded induction and the lemmas proven in the previous section:

\begin{alltt}
  Lemma binary_power_mult_ok :
    forall n a x, binary_power_mult M a x n = a * x ** n.
  Proof.
    intro n; pattern n; apply lt_wf_ind.
    {\it\color{red}(* Proof skipped *)}
\end{alltt}

Then the main theorem follows immediately:

\begin{alltt}
  Lemma binary_power_ok : forall (x:A)(n:nat), binary_power x n = x ** n.
  Proof.
    intros n x;unfold binary_power;rewrite binary_power_mult_ok;
    monoid_simpl;auto.
  Qed.  
\end{alltt}



\subsection{Discharging the Context}

It is time to close the section we opened for writing our proof of equivalence.
The theorem \texttt{binary\_power\_ok} is now provided with an universal
quantification over all the parameters of any monoid.

\begin{alltt}
End About_power.

About binary_power_ok.\it\color{red}
binary_power_ok :
forall (A : Type) (dot : A -> A -> A) (one : A) (M : Monoid dot one) 
  (x : A) (n : nat), binary_power x n = power x n

Arguments A, dot, one, M are implicit and maximally inserted
Argument scopes are [type_scope _ _ _ _ nat_scope]
binary_power_ok is opaque
Expands to: Constant Top.binary_power_ok\tt\color{black}

Check binary_power_ok 2 20.\it\color{red}
binary_power_ok 2 20
     : binary_power 2 20 = power 2 20\tt\color{black}

Let Mfib := Build_M2 1 1 1 0.

Check binary_power_ok Mfib 56.\it\color{red}

 binary_power_ok Mfib 56
     : binary_power Mfib 56 = power Mfib 56

\end{alltt}


\subsection{Subclasses}
We could prove many useful equalities in the section \texttt{about\_power}.
Nevertheless, we couldn't prove the equality $(xy)^n = x^n\,y^n$, because it is false
in general --- consider for instance the free monoid of strings, or simply
matrix multiplication. Nevertheless, this equality holds in every commutative
(a.k.a. \emph{abelian}) monoid.

Thus we say that abelian monoids form a \emph{subclass} of the class 
of monoids, and prove this equality in a context declaring an arbitrary
instance of this subclass.

Structurally, we parameterize the new class \texttt{Abelian\_Monoid}
by an arbitrary instance $M$ of \texttt{Monoid}, and add a new field
stating the commutativity of \texttt{dot}. Please keep in mind that
we declared \texttt{A}, \texttt{dot} and \texttt{one} as
\emph{generalizable variables}, hence we can use the backquote symbol here:

\begin{alltt}
Class Abelian_Monoid `(M:Monoid A dot one) := \{
  dot_comm : forall x y, dot x y = dot y x\}.
\end{alltt}

A quick look at the representation of \emph{Abelian\_Monoid} as a record type
helps us understand how this class is implemented.

  \begin{alltt}
Print Abelian_Monoid.\it\color{red}
Record Abelian_Monoid (A : Type) (dot : A -> A -> A) 
(one : A) (M : Monoid dot one) : Prop := Build_Abelian_Monoid
  \{ dot_comm : forall x y : A, dot x y = dot y x \}

For Abelian_Monoid: Arguments A, dot, one are implicit and maximally inserted
For Build_Abelian_Monoid: Arguments A, dot, one are implicit
For Abelian_Monoid: Argument scopes are [type_scope _ _ _]
For Build_Abelian_Monoid: Argument scopes are [type_scope _ _ _ _]
\end{alltt}


For building an instance of \texttt{Abelian\_Monoid}, we can start 
from \texttt{ZMult}, the monoid on \texttt{Z}, adding a proof that
integer multiplication is commutative.

\begin{alltt}
Instance ZMult_Abelian : Abelian_Monoid ZMult.
Proof.
  split. 
  exact Zmult_comm.
Defined.
\end{alltt}

We can now prove our equality by building an appropriate context.
Note that we can specify just the parameters of the monoid here in the
binder of the abelian monoid, an instance of monoid on those same
parameters is automatically generalized. Superclass parameters are
automatically generalized inside quoted binders. Again, this is simply
syntactic sugar.

\begin{alltt}
Section Power_of_dot.
  Context `\{AM:Abelian_Monoid A dot one\}.

  Theorem power_of_mult : forall n x y, 
      power (dot x y) n = dot (power x n) (power y n). 
  Proof.
    induction n;simpl.
    rewrite one_left;auto.
    intros; rewrite IHn; repeat rewrite dot_assoc.
    rewrite <- (dot_assoc x y (power x n)); rewrite (dot_comm y (power x n)).
    repeat rewrite dot_assoc;trivial.
  Qed.

End Power_of_dot.

Check power_of_mult 3 4 5.\it\color{red}
power_of_mult 3 4 5
     : power (4 * 5) 3 = power 4 3 * power 5 3
\end{alltt} 

\chapter{Relations and rewriting}

\section{Introduction: Lost in Manhattan}

Assume you are lost in Manhattan, or in any city with the same geometry:
square blocks, square blocks, and so on.

You ask some by-passer how to go to some other place, and you probably will get
an answer like that:
\begin{quote}
``go two blocks northward, then one block eastward, then three
blocks southward, and finally two blocks westward''.   
\end{quote}


You thank this kind person, and you go one block southward, then one block
westward.

If we represent such routes by lists of directions, you will consider
that the two considered routes 
\begin{itemize}
\item \texttt{North::North::East::South::South::South::West::West::nil}
\item \texttt{South::West::nil}
\end{itemize}

are not \emph{equal}  --- because the former route is much longer than the other one ---, but \emph{equivalent}, which means that they both 
lead to the same point.

Then you notice that the route \texttt{West::North::East::South::nil}
is equivalent to \texttt{nil} (which means ``don't move!'').

From both previous equivalences you want to infer simply that the long
route 
\begin{alltt}
(North::North::East::South::South::South::West::West::nil)++
(West::North::East::South::nil)
\end{alltt}
 is equivalent to \texttt{South::West::nil}.


Moreover, you can consider this equivalence as a \emph{congruence} w.r.t.
route concatenation. For instance, one can easily show that the above
routes are equivalent using a lemma stating that if 
the route \texttt{r} is equivalent to \texttt{r'}, and \texttt{s} is
equivalent to \texttt{s'} then \texttt{r++s} is equivalent to
\texttt{r'++s'}. We will say that list concatenation is
a \emph{proper} function with respect to the equivalence between routes.

The {\coq} system provides now some useful tools for considering relations
and proper functions, allowing to use the \texttt{rewrite} tactics for 
relations that are weaker than the Leibniz equality. Examples of such relations
include route-equivalence, non-canonical representation of finite sets, 
partial commutation monoids, etc.

Let us now start with our example.

\section{Data Types and Definitions}

We first load some useful modules, for representing directions, routes, and 
the discrete plane on which routers operate:


\begin{alltt}
Require Import List ZArith Bool.
Open Scope Z_scope.

Inductive direction : Type := North | East | South | West.

Definition route := list direction.

Record Point : Type :=
  \{Point_x : Z; Point_y : Z\}.

Definition Point_O := Build_Point 0 0.
\end{alltt}

\section{Route Semantics}
A route is just a ``program'' for moving from some point to another one.
The function \texttt{move} associates to any route a function from \texttt{Point} to \texttt{Point}.

\begin{alltt}
Definition translate (dx dy:Z) (P : Point) :=
  Build_Point (Point_x P + dx) (Point_y P + dy).

Fixpoint move (r:route) (P:Point) : Point :=
 match r with
 | nil => P
 | North :: r' => move r' (translate 0 1 P)
 | East :: r' => move r' (translate 1 0 P) 
 | South :: r' => move r' (translate 0 (-1) P)
 | West :: r' => move r' (translate (-1) 0 P)
 end.

Definition Point_eqb (P P':Point) :=
   Zeq_bool (Point_x P) (Point_x P') \&\&
   Zeq_bool (Point_y P) (Point_y P').

Lemma Point_eqb_correct : forall p p', Point_eqb p p' = true <->
                                       p = p'.
\it Proof skipped.\tt
Qed.
\end{alltt}

We consider that two routes are "equivalent" if they define
the same moves. For instance, the routes
\texttt{East::North::West::South::East::nil} and \texttt{East::nil}
are equivalent.

\begin{alltt}
Definition route_equiv : relation route :=
  fun r r' => forall P, move r P = move r' P.
Infix "=r=" := route_equiv (at level 70):type_scope.

Example Ex1 : East::North::West::South::East::nil =r=  East::nil.
Proof.
  intro P;destruct P;simpl.
  unfold route_equiv,translate;simpl;f_equal;ring.
Qed.
\end{alltt}

\section{On Route Equivalence}

It is easy to study some abstract properties of the relation \texttt{route\_equiv}. First, we prove that this relation is reflexive, symmetric and transitive:

\begin{alltt}
Lemma route_equiv_refl : Reflexive route_equiv.
Proof.
 intros r p;reflexivity.
Qed.

Lemma route_equiv_sym : Symmetric route_equiv.
Proof.
 intros r r' H p; symmetry;apply H.
Qed.

Lemma route_equiv_trans : Transitive route_equiv.
Proof.
 intros r r' r'' H H' p; rewrite H; apply H'.
Qed.
\end{alltt}

Note that, despite these properties, the tactics \texttt{reflexivity},
\texttt{symmetry}, and \texttt{transitivity} are not directly usable
on the type \texttt{route}.

\begin{alltt}
Goal forall r, route_equiv r r.
intro; reflexivity.\it\color{red}
Toplevel input, characters 40-51:
Error: Tactic failure: The relation route_equiv is not a 
  declared reflexive relation.
   Maybe you need to require the Setoid library.
\end{alltt}

The last error message comes from the fact that the tactic 
\texttt{reflexivity} consults a base of registered instance of a class
named \texttt{Reflexive}. There exists also classes named \texttt{Symmetric},
\texttt{Transitive}, and \texttt{Equivalence}.
This last class gathers the properties of reflexivity, symmetry and transitivity.
Thus, we can register an instance of the \texttt{Equivalence} class (and
not as a lemma as above).

\begin{alltt}
Instance route_equiv_Equiv : Equivalence  route_equiv.
Proof. split;
  [apply route_equiv_refl | 
   apply route_equiv_sym | 
   apply route_equiv_trans].
Qed.
\end{alltt}

The tactics \texttt{reflexivity}, etc.  can now be applied to the 
relation \texttt{route\_equiv}.

\begin{alltt}
Goal forall r, r =r= r.
Proof. intro; reflexivity. Qed.
\end{alltt}

Note that it is possible to consider relations that are not equivalence 
relations, but are only transitive, or reflexive and transitive, etc. by
using the right type classes.

\section{Proper Functions}

Since routes are represented as lists of directions, we wish to prove 
some route equivalences by composition of already proven equivalences,
using some lemmas on consing and appending routes.

For instance, we can prove that if we add the same direction in front of
two equivalent routes, the routes we obtain are still equivalent:

\begin{alltt}
Lemma route_cons : forall r r' d, r =r= r' -> d::r =r= d::r'.
Proof.
 intros r r' d H P;destruct d;simpl;rewrite H;reflexivity.
Qed.
\end{alltt}

This lemma can be applied  to re-use our previous example \texttt{Ex1}.

\begin{alltt}
Goal (South::East::North::West::South::East::nil) =r= (South::East::nil).
Proof. apply route_cons;apply Ex1. Qed.
\end{alltt}

Note that the proof of \texttt{route\_cons} contains a case analysis on $d$. Let us try to have a more
direct proof:

\begin{alltt}
Lemma route_cons' : forall r r' d, r =r= r' -> d::r =r= d::r'.
Proof.
  intros r r' d H;rewrite H.\it\color{red}

Toplevel input, characters 80-89:
Error: Found no subterm matching "move r ?17869" in the current goal. \tt\color{black}
Abort. 
\end{alltt}

What we really need is to tell to the \texttt{rewrite} tactic how to
use \texttt{route\_cons} for using an equivalence \texttt{$r$ =r= $r'$}
for replacing $r$ with $r'$ in a term of the form \texttt{cons $d$ $r$}
for proving directly the equivalence \texttt{cons $d$ $r$ =r= cons $d$ $r'$}.
In other words, we say that \texttt{cons $d$} is \emph{proper} w.-r.-t.
the relation \texttt{route\_equiv}.

In {\coq} this fact can be declared as an instance of:
\begin{alltt}
   Proper (route\_equiv ==> route\_equiv) (cons d)
\end{alltt}

This notation, which requires the library \texttt{Morphisms} to be loaded,
expresses that if two routes $r$ and $r'$ are related through 
\texttt{route\_equiv}, then the routes \texttt{$d$::r} and \texttt{$d$::r'}
are also related by \texttt{route\_equiv}.

In the notation used by \texttt{Proper} the first occurrence of
\texttt{route\_equiv} refers to the arguments of \texttt{cons $d$}, and the second one 
to the result of this consing. The user must require and import 
the \texttt{Morphisms} module to use this notation.
\begin{alltt}
Require Import Morphisms.

Instance cons_route_Proper (d:direction) : 
    Proper (route_equiv ==> route_equiv) (cons d).
Proof.
  intros r r' H ;apply route_cons;assumption.
Qed.
\end{alltt}

We can now use \texttt{rewrite} to replace some 
term by an equivalent one, provided the context is made by applications of
one or several functions of the form \texttt{cons $d$}:

\begin{alltt}
Goal forall r r', r =r= r' -> South::West::r =r= South::West::r'.
Proof.
  intros r r' H. \it\color{red}

1 subgoal
  
  r : route
  r' : route
  H : r =r= r'
  ============================
   South :: West :: r =r= South :: West :: r'
\end{alltt}

Since the context of the  
variable \texttt{r} is composed of applications of proper function
 calls, the tactic \texttt{rewrite H} can be used to replace the occurrence of
\texttt{r} by \texttt{r'}:

\begin{alltt}
  rewrite H;reflexivity.
Qed.
\end{alltt}

\section{Some Other instances of \texttt{Proper}}


We prove also that the function \texttt{move} 
is proper with respect to route equivalence and Leibniz equality on points:


\begin{alltt}
Instance move_Proper : Proper (route_equiv ==> eq ==> eq) move. 
Proof.
  intros r r' Hr_r' p q Hpq. rewrite Hpq; apply Hr_r'.
Qed.
\end{alltt}


Let us prove now that list concatenation is proper w.r.t.
\texttt{route\_equiv}.

First, a technical lemma, that relates \texttt{move} and list concatenation:


\begin{alltt}
Lemma route_compose : forall r r' P, move (r++r') P = move r' (move r P).
Proof.
  induction r as [|d s IHs]; simpl;
    [auto | destruct d; intros;rewrite IHs;auto].
Qed.
\end{alltt}

Then the proof itself:
\begin{alltt}
Instance app_route_Proper : 
  Proper (route_equiv ==> route_equiv ==> route_equiv) (@app direction).
Proof.
  intros r r' H r'' r''' H' P.
  repeat rewrite route_compose.\it\color{red}

1 subgoal
  
  r : route
  r' : route
  H : r =r= r'
  r'' : route
  r''' : route
  H' : r'' =r= r'''
  P : Point
  ============================
    move r'' (move r P) = move r''' (move r' P)\tt\color{black}

  now rewrite H, H'.
Qed.
\end{alltt}

We are now able to compose proofs of route equivalence:

\begin{alltt}
Example Ex2 : forall r, North::East::South::West::r =r= r.
Proof. intros r P;destruct P;simpl. 
  unfold route_equiv, translate;simpl;do 2 f_equal;ring.
Qed.

Example Ex3 : forall r r', r =r= r' -> 
                North::East::South::West::r =r= r'.
Proof. intros r r' H. now rewrite Ex2. Qed.

Example Ex4 : forall r r',  r++ North::East::South::West::r' =r= r++r'.
Proof. intros r r'. now rewrite Ex2. Qed.
\end{alltt}

This generalized rewriting principle applies to equivalence relations,
but also to e.g. order relations and can be used to rewrite under
binders, where the usual \texttt{rewrite} tactic fails. It is presented
in more detail in the reference manual and the article by \cite{sozeauJFR09}.


\subsection*{A non proper function}
It is easy to give an example of a non-proper function. Two routes may be
equivalent, but have different lengths.

\begin{alltt}
Example length_not_Proper : ~Proper (route_equiv ==> @eq nat) (@length _).
Proof. intro H.
  generalize (H (North::South::nil) nil);simpl;intro H0.
  discriminate H0.
  intro P;destruct P; simpl;unfold translate; simpl;f_equal;simpl;ring.
Qed
\end{alltt}

\section{Deciding Route Equivalence}
Proofs of examples like \texttt{Ex2} may seem too complex for proving
that some routes are equivalent. It is better to define a simple boolean
function for deciding equivalence, and use reflection in case we have
closed terms of type $route$.

We first define a boolean function :
\begin{alltt}
Definition route_eqb r r' : bool :=
   Point_eqb (move r Point_O) (move r' Point_O).
\end{alltt}

Some technical lemmas proved in \texttt{Lost\_in\_NY} lead us to prove 
the following equivalence, that allows us to prove some path equivalences
through a simple computation:

\begin{alltt}
Lemma route_equiv_equivb : forall r r', route_equiv r r' <->
                                        route_eqb r r' = true.

Ltac route_eq_tac := rewrite route_equiv_equivb;reflexivity.

Example Ex1' : route_equiv (East::North::West::South::East::nil) (East::nil).
Proof. route_eq_tac. Qed.
\end{alltt}

\section{Monoids and Setoids}
We would like to prove that route concatenation is associative, and admits
the empty route \texttt{nil} as a neutral element, making the type 
\texttt{route} the carrier of some monoid.
 
The \texttt{Monoid} class type defined in Sect.~\ref{Monoid-def} cannot be
used for this purpose, since the properties of associativity and
being a neutral element are defined in terms of Leibniz equality.

We give below a definition of a new class \texttt{EMonoid} parameterized by
an equivalence relation:

\begin{alltt}
Class EMonoid {A:Type}(E_eq :relation A)(dot : A->A->A)(one : A):=\{
  E_rel :> Equivalence E_eq; 
  dot_proper :> Proper (E_eq ==> E_eq ==> E_eq) dot; 
  E_dot_assoc : forall x y z : A, E_eq (dot x (dot y z)) (dot (dot x y) z);
  E_one_left : forall x, E_eq (dot one x) x;
  E_one_right : forall x, E_eq (dot x one) x \}.

Fixpoint Epower `\{M: EMonoid A E_eq dot one\}(a:A)(n:nat) :=
  match n with 
  | 0%nat => one
  | S p => dot a (Epower a p)
  end.
\end{alltt}

We register an instance of \texttt{EMonoid}:

\begin{alltt}
Instance Route : EMonoid route_equiv (@app _) nil .
Proof. split.
  apply route_equiv_Equiv.
  apply app_route_Proper.
  intros x y z P;repeat rewrite route_compose; trivial.
  intros x P;repeat rewrite route_compose; trivial.
  intros x P;repeat rewrite route_compose; trivial.
Qed.
\end{alltt}

We can readily compute exponentiation of routes using this monoid:
\begin{alltt}
Goal forall n, Epower (South::North::nil) n =r= nil.
Proof. induction n; simpl.
  reflexivity.
  rewrite IHn.
  route_eq_tac.
Qed.
\end{alltt}


\paragraph{Exercise}
It seems that route concatenation is commutative as far as route equivalence is
concerned. Is this true? In this case, define a subclass of \texttt{EMonoid}
that handles commutativity, and build an instance of this class for route
concatenation and equivalence.

\section{Advanced Features of Type Classes}

\subsection{Alternate definitions of the class \texttt{Monoid}}
\label{alternate}
Note that \texttt{A}, \texttt{dot} and \texttt{one} are \emph{parameters}
of this definition, whilst they could have been defined as \emph{fields}
of a structure.

The following variant is correct too:
\begin{alltt}
Class Monoid'  : Type := \{
  carrier: Type;
  dot : carrier -> carrier -> carrier; 
  one : carrier;
  dot_assoc : forall x y z:carrier, dot x (dot y z)= dot (dot x y) z;
  one_left : forall x, dot one x = x;
  one_right : forall x, dot x one = x\}.
\end{alltt}

However, there is a flaw in the definition of \texttt{Monoid'}: if for some
reason one needs to consider two monoid structures ${\cal M}$ and
${\cal M'}$ on the same carrier, the sharing of the considered carrier would
be very clumsy to express and to reason about.

\begin{alltt}
Section TwoMonoids.
 Variables M M' : Monoid'.
 Hypothesis same_carrier : @carrier M = @carrier M'.
 Hypothesis same_one : @one M = @one M'.\it\color{red}
Toplevel input, characters 52-59:
Error:
In environment
M : Monoid'
M' : Monoid'
same_carrier : carrier = carrier
The term "one" has type "carrier" while it is expected to have type
 "carrier".
\end{alltt}

A possible solution would be using JM equality, but we don't get the
simplicity of the approach of Sect~\ref{on-monoids}.
 \begin{alltt}
 Require Import JMeq.
 Hypothesis same_one : JMeq (@one M) (@one M').
\end{alltt}

Any definition using both monoids in the same expression will be filled
with coercions between the propositionally equal but not definitionally
equal carriers and operations, so this is an unworkable solution. 

A second variant could be to consider the carrier $A$ as the only parameter of
the definition and to leave the binary operation and neutral element as fields
of the class:
\begin{alltt}
Class Monoid' (A:Type) := \{
 dot : A ->  A -> A;
 one : A;
 dot_assoc : forall x y z:A, dot x (dot y z)= dot (dot x y) z;
 one_left : forall x, dot one x = x;
 one_right : forall x, dot x one = x\}.
\end{alltt}

It is possible to define the subclass of abelian monoids with this representation:
\begin{alltt}
Class AbelianMonoid' (A:Type) := \{
 underlying_monoid :> Monoid' A;
 dot_comm : forall x y, dot x y = dot y x \}.

Section Foo.
  Variable A : Type.
  Context (AM : AbelianMonoid' A).

  Goal forall x y z, dot x (dot y z) = dot y (dot x z).
  Proof. intros x y z.
    repeat rewrite dot_assoc.
    now rewrite (dot_comm x y).
  Qed.
End Foo.

Require Import Arith.

Instance Mnat : Monoid' nat.
Proof. split with plus 0;auto with arith. Defined.

Instance AMnat : AbelianMonoid' nat.
Proof. split with Mnat. unfold dot;auto with arith. Defined.

Goal dot 3 4 = 7.
Proof. reflexivity. Qed.
\end{alltt}

Nevertheless,  the reader will find in a paper by~\citet{math-classes}
arguments for disregarding this variant:
In short, some clumsiness would appear if we want to make \texttt{Monoid'}
a member of new classes like \texttt{Group}, \texttt{Ring}, etc.



\subsection{Operational Type Classes}
Let us come back to the definitions of section Sect.~\ref{on-monoids}
Let us assume we wish to write functions and mathematical statements
with a syntax as close as possible to the standard mathematical notation. 
For instance, nested applications of the \texttt{dot} operation should
be written with the help of an infix operator.

The \texttt{Infix} command cannot be used directly, since \texttt{dot}
is not declared as a global constant. 

\begin{alltt}
Infix "*" := dot (at level 50, left associativity):M_scope. \it 
Error: The reference dot was not found in the current environment.
\end{alltt}
 A solution from ibid. consists in declaring a \emph{singleton type class} for representing binary operators:

\begin{alltt}
Class monoid_binop (A:Type) := monoid_op : A -> A -> A.   
\end{alltt}

\emph{Nota: }Unlike multi-field class types, \texttt{monoid\_op} is not a constructor,
but a transparent constant such that \texttt{monoid\_op $f$} can be
$\delta\beta$-reduced into $f$. 

It is now possible to declare an infix notation:
\begin{alltt}
Delimit Scope M_scope with M.
Infix "*" := monoid_op: M_scope.
Open Scope M_scope.
\end{alltt}

\label{Monoid-def}
We can now give a new definition of \texttt{Monoid}, using the type
\texttt{monoid\_binop $A$} instead of $A\arrow A \arrow A$, and the
infix notation \texttt{x * y} instead of \texttt{monoid\_op x y}:

\begin{alltt}
Class Monoid (A:Type)(dot : monoid_binop  A)(one : A) : Type := {
  dot_assoc : forall x y z : A, x * (y * z)=  x * y * z;
  one_left : forall x, one * x = x;
  one_right : forall x,  x * one = x}.
\end{alltt}

For building monoids like \texttt{ZMult} or \texttt{M2\_Monoid}, we 
first declare instances of \texttt{monoid\_binop}.

\begin{alltt}
Instance Zmult_op : monoid_binop Z := Zmult. 

Instance ZMult : Monoid Zmult_op 1.
Proof. split;intros; unfold Zmult_op, monoid_op; ring. Defined.

Section matrices. 
  Variables (A:Type)
            (zero one : A) 
            (plus mult minus : A -> A -> A)
            (sym : A -> A).
  {\color{blue}(* Definitions skipped, see section \ref{matring} *) }

  Global Instance M2_mult_op : monoid_binop M2 := M2_mult.

  Global Instance M2_Monoid : Monoid M2_mult_op Id2. 
  Proof. (* Proof skipped *) Defined.
End matrices.
\end{alltt}

For dealing with matrices on \texttt{Z}, we may instantiate the parameters
\texttt{A}, \texttt{zero}, etc.

\begin{alltt}
Instance M2_Z_op : monoid_binop (M2 Z) := M2_mult Zplus Zmult . 
Instance M2_mono_Z : Monoid (M2_mult_op _ _)  (Id2 _ _):=  M2_Monoid Zth.  

Compute ((Build_M2 1 1 1 0) * (Build_M2 1 1 1 0))%M.\it\color{red}
     = \{| c00 := 2; c01 := 1; c10 := 1; c11 := 1 |\}\tt\color{black}
     : M2 Z
Compute ((Id2 0 1) * (Id2 0 1))%M.\it\color{red}
   = \{| c00 := 1; c01 := 0; c10 := 0; c11 := 1 |\}\tt\color{black}
     : M2 Z
\end{alltt}



It is now easy to use the infix notation \texttt{*} in the definition of
functions like \texttt{power}:
\begin{alltt}
Generalizable Variables A dot one.

Fixpoint power `\{M : Monoid A dot one\}(a:A)(n:nat) :=
  match n with 
  | 0\%nat => one
  | S p => (a * power a p)\%M
  end.

Infix "**" := power (at level 30, no associativity):M_scope.

Compute  (2 ** 5) ** 2.\it
= 1024 :Z\tt
Compute (Build_M2  1 1 1  0) ** 40.\it
    = \{|
       c00 := 165580141;
       c01 := 102334155;
       c10 := 102334155;
       c11 := 63245986 |\}
     : M2 Z
\end{alltt}


All the techniques we used in Sect.~\ref{equiv-proof} can be applied with
operational type classes. The main section of this proof is as follows:

\begin{alltt}
Section About_power.
  Context `(M:Monoid A).
  Open Scope M_scope.
  
  Ltac monoid_rw :=
    rewrite one_left || rewrite one_right || rewrite dot_assoc.
  Ltac monoid_simpl := repeat monoid_rw.

  Lemma power_x_plus : forall x n p, x ** (n + p) = x ** n * x ** p.
  Proof. (* Proof skipped *) Qed.
End About_power.
\end{alltt}

Notice that, when the section is closed, theorem statements keep the
notations of \texttt{M\_scope}:

\begin{alltt}
Open Scope M_scope.
About power_of_power.\it\color{red}

power_of_power :
forall (A : Type) (dot : monoid_binop A) (one : A) 
  (M : Monoid dot one) (x : A) (n p : nat), 
        (x ** n) ** p = x ** (p * n)
\end{alltt}


If we want to consider monoids w.r.t. some equivalence relation, 
it is possible to associate an operational type class to the considered relation:

\begin{alltt}
Require Import Setoid Morphisms.

Class Equiv A := equiv : relation A.
Infix "==" := equiv (at level 70):type_scope.

Open Scope M_scope.
Class EMonoid (A:Type)(E_eq :Equiv A)
              (E_dot : monoid_binop A)(E_one : A):= \{
  E_rel :> Equivalence equiv; 
  dot_proper :> Proper (equiv ==> equiv ==> equiv) E_dot; 
  E_dot_assoc : forall x y z:A,
      x * (y * z) == x * y * z;
  E_one_left : forall x, E_one * x == x;
  E_one_right : forall x, x * E_one == x\}.
\end{alltt}

The overloaded \texttt{==} notation will refer to the appropriate
equivalence relation on the type of the arguments. One can develop in
this fashion a hierarchy of structures. In the following we define
the structure of semirings starting from a refinement of \texttt{EMonoid}.

\begin{alltt}
(* Overloaded notations *)
Class RingOne A := ring_one : A.
Class RingZero A := ring_zero : A.
Class RingPlus A := ring_plus :> monoid_binop A.
Class RingMult A := ring_mult :> monoid_binop A.
Infix "+" := ring_plus.
Infix "*" := ring_mult.
Notation "0" := ring_zero.
Notation "1" := ring_one.

Typeclasses Transparent RingPlus RingMult RingOne RingZero.

Class Distribute `\{Equiv A\} (f g: A -> A -> A): Prop :=
  \{ distribute_l a b c: f a (g b c) == g (f a b) (f a c)
  ; distribute_r a b c: f (g a b) c == g (f a c) (f b c) \}.

Class Commutative \{A B\} `\{Equiv B\} (m: A -> A -> B): Prop := 
  commutativity x y : m x y == m y x.

Class Absorb {A} `\{Equiv A\} (m: A -> A -> A) (x : A) : Prop := 
  \{ absorb_l c : m x c == x ;
    absorb_r c : m c x == x \}.

Class ECommutativeMonoid `(Equiv A) (E_dot : monoid_binop A) (E_one : A):=
  \{ e_commmonoid_monoid :> EMonoid equiv E_dot E_one;
    e_commmonoid_commutative :> Commutative E_dot \}.

Class ESemiRing (A:Type) (E_eq :Equiv A) 
  (E_plus : RingPlus A) (E_zero : RingZero A)
  (E_mult : RingMult A) (E_one : RingOne A):=
  \{ add_monoid :> ECommutativeMonoid equiv ring_plus ring_zero ;
    mul_monoid :> EMonoid equiv ring_mult ring_one ;
    ering_dist :> Distribute ring_mult ring_plus ;
    ering_0_mul :> Absorb ring_mult 0 \}.
\end{alltt}

Note that we use a kind of multiple inheritance here, as a semiring contains two
monoids, one for addition and one for multiplication, that are related
by distributivity and absorbtion laws. To distinguish between the
corresponding monoid operations, we introduce the new operational type
classes \texttt{Ring}*. These classes are declared \texttt{Transparent}
for typeclass resolution, so that their expansion to
\texttt{monoid\_binop} can be used freely during conversion: they are
just abbreviations used for overloading notations.

We also introduce classes for the standard properties of operations like
commutativity, distributivity etc... to be able to refer to them generically.

We can now develop some generic theory on semirings using the overloaded
lemmas about distributivity or the constituent monoids. Resolution
automatically goes through the \texttt{ESemiRing} structure to find
proofs regarding the underlying monoids.

\begin{alltt}
Section SemiRingTheory.
  Context `\{ESemiRing A\}.

  Definition ringtwo := 1 + 1.
  Lemma ringtwomult : forall x : A, ringtwo * x == x + x.
  Proof.
    intros. unfold ringtwo.
    rewrite distribute_r.
    now rewrite (E_one_left x).
  Qed.

End SemiRingTheory.
\end{alltt}

\subsection{Instance Resolution}
Let us consider again the term written ``\,\applis{binary\_power\;2\; 55}\,''.
This term --- with all its arguments explicited --- is ``\,\applis{@binary\_power Z Zmult\_op 1\;2\;55}\,''. The implicit arguments
\texttt{Z}, \texttt{Zmult\_op} and \texttt{1} have been inferred 
from the type \texttt{Z} of \texttt{2}, looking at the instances
of \texttt{Monoid} that were compatible with the instanciation
\texttt{A:=Z}. 
 
% \textbf{MS: C'est un bug de Coq de ne pas respecter la specification de
%   la recherche d'instance. La specification est: on choisit l'instance 
%   la plus rÈcente parmis les instances applicables de la prioritÈ la
%   plus faible. Dans la suite, Zmult\_op est choisit tandis que Zplus\_op
%   devrait l'Ítre (mÍme prioritÈ, mais Zplus\_op est plus rÈcente). La
%   gestion des prioritÈs pour indiquer la prÈfÈrence fonctionne en revanche.}

If we consider another instance of \texttt{Monoid} with the same 
domain, the instance resolution mechanism, which relies on \texttt{auto},
may not respect the user's intuition. 

For instance, let us temporarily consider another monoid on \texttt{Z}.

\begin{alltt}
Section Z_additive.
  Instance Z_plus_op : monoid_binop Z := Zplus.

  Instance ZAdd : Monoid Z_plus_op 0. 
  Proof. split;intros;unfold Z_plus_op, mon_op;ring. Defined.
\end{alltt}

Unfortunately, two instances of \texttt{monoid\_binop} are linked to the type 
\texttt{Z}. Thus, a notation like \texttt{2 * 5} looks ambiguous: 

In the interaction below, the instance \texttt{Z\_plus\_op} is preferred
over \texttt{Z\_mult\_op}, as it was declared the latest.
\begin{alltt}
Compute 2 * 5.\it
     = 7
     : Z
\end{alltt}

To force an unambiguous interpretation regardless of the order of
declarations, one solution is to add to the instance declaration a
\emph{cost}, which is a natural number: the lowest the cost, the
higher is the priority.  Let us consider the following two instance
declarations:

\begin{alltt}
Instance Zmult_op : monoid_binop Z | 1 := Zmult.
Instance Zplus_op : monoid_binop Z | 2 := Zplus.

Compute (2 * 5)%M.\it\color{red}
= 10 : nat
\end{alltt}

Now the highest priority (lowest cost) instance is selected.





\chapter*{Conclusion}

This concludes our tour of type classes and generalized rewriting. For
more information on classes, one can consult the lecture notes by \citet
{sozeau.Coq/classes/JFLA12} which develops other examples of
use. \citet{math-classes} develop a large hierarchy of structures
using classes, including pervasive use of generalized rewriting. 
The current version of their library is available online at:
\url{https://github.com/math-classes}. 
The library on categories and categorical syntax and semantics by
Benedikt Ahrens~\cite{Ahrens}, uses heavily type classes and is worth reading.


Generalized rewriting \citep{sozeauJFR09} is an example of the use of
type classes to develop a \emph{generic} tactic that can apply to
arbitrary user constants as long as ``Proper'' instances are declared on
them. As we've seen, the very nature of type classes is to do proof
search at typechecking-time. They are hence very fit to implement
generic tactics that depend on some properties of the terms fed to them,
for example associativity or commutativity of functions in the work of
\citet{DBLP:journals/corr/abs-1105-4537}.
They can also be used to concisely specify domain-specific proof
automation, solving side conditions for a particular class of problems
as in \cite{DBLP:conf/icfp/GonthierZND11}. These papers provide more
advanced examples and design patterns for working with type classes.

\bibliography{biblio,morebiblio}
\bibliographystyle{plainnat}

\end{document}
